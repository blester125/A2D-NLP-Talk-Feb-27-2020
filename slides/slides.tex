\documentclass{beamer}

\usefonttheme[onlymath]{serif}
\usepackage[utf8]{inputenc}
\usepackage{amsmath}
\usepackage{array}
\usepackage{graphicx}
\usepackage{mathtools}
\usepackage{minted}
\usepackage{hyperref}
\hypersetup{
    colorlinks=true,
    linkcolor=blue,
}
\usemintedstyle{manni}
\newminted{python}{fontsize=\footnotesize}

\usetheme{Pittsburgh}

\usepackage{pgfpages}
\setbeamertemplate{note page}{\pagecolor{yellow!5}\insertnote}
\setbeameroption{show notes on second screen=right}

\usepackage{tikz}
\usetikzlibrary{fit}
\tikzset{%
  highlight/.style={rectangle,rounded corners,fill=red!15,draw,
    fill opacity=0.5,thick,inner sep=0pt}
}
\newcommand{\tikzmark}[2]{\tikz[overlay,remember picture,
  baseline=(#1.base)] \node (#1) {#2};}
%
\newcommand{\Highlight}[1][submatrix]{%
    \tikz[overlay,remember picture]{
    \node[highlight,fit=(left.north west) (right.south east)] (#1) {};}
}

\title{Your Neural Network for NLP is Probably Wrong}
\subtitle{Why You Need to Mask More Than You Think}
\author{Brian Lester}
\institute{Interactions}
\date{February, 27, 2020}

\def\R{\mathbb{R}}
\def\N{\mathbb{N}}

\begin{document}

\frame{\titlepage}
\begin{section}{Bio}

    \begin{frame}
        \frametitle{Me}
        \begin{itemize}
            \item Work on NLP at Interactions
            \item My specialization is in Deep Learning
            \item Maintain \href{https://github.com/dpressel/mead-baseline}{Mead-Baseline}
        \end{itemize}
        \note{
            \begin{itemize}
                \item Why you should listen to me
                \item We use NLP to ``facilitate customer care interactions'' aka build better chat bots
                \item We use DL for most models so I have a lot of experience
                \item I help maintain Mead-Baseline, our open-source Deep Learning modeling software
                \item I have batched some complex operations
                \begin{itemize}
                    \item Batched the CRF
                    \item Batched Beam Search
                \end{itemize}
                \item I have tracked down a lot of hidden batch instability problems
                \item I am trying to get more involved with research so if anyone needs help with some idea, especially if
                    the implementation is tricky, hit me up.
            \end{itemize}
        }
    \end{frame}

\end{section} % Bio

\begin{section}{Motivation}
    \begin{subsection}{What is batching?}
        \begin{frame}
            \frametitle{Binary Logistic Regression}
            \begin{align*}
                n &\coloneqq \text{number of features} \\
                c &\coloneqq \text{number of class} \coloneqq 1 \\
                f &\in \R ^ {n} \\
                w &\in \R ^ {n} \\
                s &= \sum_{i=0}^{n} f_i * w_i \\
            \end{align*}

        \end{frame}

\begin{frame}[fragile]
    \frametitle{Binary Logistic Regression}
    \begin{pythoncode}
                        s = np.dot(f, w)
    \end{pythoncode}
\end{frame}

        \begin{frame}
            \frametitle{Binary Logistic Regression}
            \begin{align*}
                f &= \left[ \begin{array}{*6{c}}
                    &\tikzmark{left}{1}\tikzmark{right} & 2 & 3 & 4 &
                    \end{array}
                    \right] \\
                \Highlight[first]
                w &= \left[ \begin{array}{*6{c}}
                    &\tikzmark{left}{5}\tikzmark{right} & 6 & 7 & 8 &
                    \end{array}
                    \right] \\
                \Highlight[second]
                s &= 5
            \end{align*}
        \end{frame}

        \begin{frame}
            \frametitle{Binary Logistic Regression}
            \begin{align*}
                f &= \left[ \begin{array}{*6{c}}
                    & 1 & \tikzmark{left}{2} \tikzmark{right}{} & 3 & 4 &
                    \end{array}
                    \right] \\
                \Highlight[first]
                w &= \left[ \begin{array}{*6{c}}
                    & 5 & \tikzmark{left}{6} \tikzmark{right}{} & 7 & 8 &
                    \end{array}
                    \right] \\
                \Highlight[second]
                s &= 17
            \end{align*}
        \end{frame}

        \begin{frame}
            \frametitle{Binary Logistic Regression}
            \begin{align*}
                f &= \left[ \begin{array}{*6{c}}
                    & 1 & 2 & 3 & 4 &
                    \end{array}
                    \right] \\
                w &= \left[ \begin{array}{*6{c}}
                    & 5 & 6 & 7 & 8 &
                    \end{array}
                    \right] \\
                s &= 70
            \end{align*}
        \end{frame}

        \begin{frame}
            \frametitle{Multi-class Logistic Regression}
            \begin{align*}
                n &\coloneqq \text{number of features} \\
                c &\coloneqq \text{number of class} \\
                f &\in \R ^ {n} \\
                W &\in \R ^ {n \text{ x } c} \\
                s &\in \R ^ {c} \\
                \forall_{j \in c}\hspace{2pt} s_j &= \sum_{i=0}^{n} f_i * W_{ij} \\
            \end{align*}

        \end{frame}

\begin{frame}[fragile]
    \frametitle{Multi-class Logistic Regression}
    \begin{pythoncode}
                    s = []
                    for w in W.T:
                        s.append(np.dot(f, w))
                    s = np.array(s)
    \end{pythoncode}
\end{frame}

\begin{frame}[fragile]
    \frametitle{Multi-class Logistic Regression}
    \begin{pythoncode}
                    s = np.dot(f, W)
    \end{pythoncode}
\end{frame}

        \begin{frame}
            \frametitle{Multi-class Logistic Regression}
            \begin{align*}
                f &= \left[ \begin{array}{*6{c}}
                    & \tikzmark{left}{1}\tikzmark{right}{} & 2 & 3 & 4 &
                    \end{array}
                    \right] \\
                \Highlight[first]
                W &= \left[ \begin{array}{*4{c}}
                    & \tikzmark{left}{5}\tikzmark{right}{} & 9  & \\
                    & 6 & 10 & \\
                    & 7 & 11 & \\
                    & 8 & 12 &
                    \end{array}
                    \right] \\
                \Highlight[second]
                s &= \left[ \begin{array}{*4{c}}
                    & \tikzmark{left}{5} \tikzmark{right}{} & &
                    \end{array}
                    \right] \\
                \Highlight[third]
            \end{align*}
        \end{frame}

        \begin{frame}
            \frametitle{Multi-class Logistic Regression}
            \begin{align*}
                f &= \left[ \begin{array}{*6{c}}
                    & 1 & \tikzmark{left}{2} \tikzmark{right}{} & 3 & 4 &
                    \end{array}
                    \right] \\
                \Highlight[first]
                W &= \left[ \begin{array}{*4{c}}
                    & 5 & 9  & \\
                    & \tikzmark{left}{6}\tikzmark{right}{} & 10 & \\
                    & 7 & 11 & \\
                    & 8 & 12 &
                    \end{array}
                    \right] \\
                \Highlight[second]
                s &= \left[ \begin{array}{*4{c}}
                    & \tikzmark{left}{17} \tikzmark{right}{} & &
                    \end{array}
                    \right] \\
                \Highlight[thrid]
            \end{align*}
        \end{frame}


        \begin{frame}
            \frametitle{Multi-class Logistic Regression}
            \begin{align*}
                f &= \left[ \begin{array}{*6{c}}
                    & \tikzmark{left}{1} & 2 & 3 & \tikzmark{right}{4} &
                    \end{array}
                    \right] \\
                \Highlight[first]
                W &= \left[ \begin{array}{*4{c}}
                    & \tikzmark{left}{5} & 9  & \\
                    & 6 & 10 & \\
                    & 7 & 11 & \\
                    & \tikzmark{right}{8} & 12 &
                    \end{array}
                    \right] \\
                \Highlight[second]
                s &= \left[ \begin{array}{*4{c}}
                    & \tikzmark{left}{70}\tikzmark{right}{} & &
                    \end{array}
                    \right] \\
                \Highlight[third]
            \end{align*}
        \end{frame}

        \begin{frame}
            \frametitle{Multi-class Logistic Regression}
            \begin{align*}
                f &= \left[ \begin{array}{*6{c}}
                    & \tikzmark{left}{1}\tikzmark{right}{} & 2 & 3 & 4 &
                    \end{array}
                    \right] \\
                    \Highlight[first]
                W &= \left[ \begin{array}{*4{c}}
                    & 5 & \tikzmark{left}{9}\tikzmark{right}{}  & \\
                    & 6 & 10 & \\
                    & 7 & 11 & \\
                    & 8 & 12 &
                    \end{array}
                    \right] \\
                    \Highlight[second]
                s &= \left[ \begin{array}{*4{c}}
                    & 70 & \tikzmark{left}{9}\tikzmark{right}{} &
                    \end{array}
                    \right] \\
                    \Highlight[third]
            \end{align*}
        \end{frame}

        \begin{frame}
            \frametitle{Multi-class Logistic Regression}
            \begin{align*}
                f &= \left[ \begin{array}{*6{c}}
                    & 1 & \tikzmark{left}{2}\tikzmark{right}{} & 3 & 4 &
                    \end{array}
                    \right] \\
                \Highlight[first]
                W &= \left[ \begin{array}{*4{c}}
                    & 5 & 9 & \\
                    & 6 & \tikzmark{left}{10}\tikzmark{right}{}  & \\
                    & 7 & 11 & \\
                    & 8 & 12 &
                    \end{array}
                    \right] \\
                \Highlight[second]
                s &= \left[ \begin{array}{*4{c}}
                    & 70 & \tikzmark{left}{9}\tikzmark{right}{} &
                    \end{array}
                    \right] \\
                \Highlight[thrid]
            \end{align*}
        \end{frame}

        \begin{frame}
            \frametitle{Multi-class Logistic Regression}
            \begin{align*}
                f &= \left[ \begin{array}{*6{c}}
                    & \tikzmark{left}{1} & 2 & 3 & \tikzmark{right}{4} &
                    \end{array}
                    \right] \\
                \Highlight[first]
                W &= \left[ \begin{array}{*4{c}}
                    & 5 & \tikzmark{left}{9}  & \\
                    & 6 & 10 & \\
                    & 7 & 11 & \\
                    & 8 & \tikzmark{right}{12} &
                    \end{array}
                    \right] \\
                \Highlight[second]
                s &= \left[ \begin{array}{*4{c}}
                    & 70 & \tikzmark{left}{9}\tikzmark{right}{} &
                    \end{array}
                    \right] \\
                \Highlight[third]
            \end{align*}
        \end{frame}


        \begin{frame}
            \frametitle{Batched Multi-class Logistic Regression}
            \begin{align*}
                n &\coloneqq \text{number of features} \\
                c &\coloneqq \text{number of class} \\
                b &\coloneqq \text{number of examples} \\
                F &\in \R ^ {b \text{ x } n} \\
                W &\in \R ^ {n \text{ x } c} \\
                S &\in \R ^ {b \text{ x } c} \\
                \forall_{k \in b} \hspace{2pt} \forall_{j \in c} \hspace{2pt} S_{kj} &= \sum_{i=0}^{n} F_{ki} * W_{ij} \\
            \end{align*}

        \end{frame}

\begin{frame}[fragile]
    \frametitle{Batched Multi-class Logistic Regression}
    \begin{pythoncode}
                S = []
                for f in F:
                    s = []
                    for w in W.T:
                        s.append(np.dot(f, w))
                    S.append(s)
                S = np.array(S)
    \end{pythoncode}
\end{frame}

\begin{frame}[fragile]
    \frametitle{Batched Multi-class Logistic Regression}
    \begin{pythoncode}
                S = []
                for f in F:
                    S.append(np.dot(f, W))
                S = np.stack(S)
    \end{pythoncode}
\end{frame}

\begin{frame}[fragile]
    \frametitle{Batched Multi-class Logistic Regression}
    \begin{pythoncode}
                S = np.dot(F, W)
    \end{pythoncode}
\end{frame}

        \begin{frame}
            \frametitle{Batched Multi-class Logistic Regression}
            \begin{align*}
                f &= \left[ \begin{array}{*6{c}}
                    & \tikzmark{left}{1}\tikzmark{right}{} & 2 & 3 & 4 & \\
                    & 13 & 14 & 15 & 16 &
                    \end{array}
                    \right] \\
                \Highlight[first]
                W &= \left[ \begin{array}{*4{c}}
                    & \tikzmark{left}{5}\tikzmark{right}{} & 9  & \\
                    & 6 & 10 & \\
                    & 7 & 11 & \\
                    & 8 & 12 &
                    \end{array}
                    \right] \\
                \Highlight[second]
                s &= \left[ \begin{array}{*4{c}}
                    & \tikzmark{left}{5} \tikzmark{right}{} & & \\
                    & & &
                    \end{array}
                    \right] \\
                \Highlight[third]
            \end{align*}
        \end{frame}

        \begin{frame}
            \frametitle{Batched Multi-class Logistic Regression}
            \begin{align*}
                f &= \left[ \begin{array}{*6{c}}
                    & \tikzmark{left}{1} & 2 & 3 & \tikzmark{right}{4} & \\
                    & 13 & 14 & 15 & 16 &
                    \end{array}
                    \right] \\
                \Highlight[first]
                W &= \left[ \begin{array}{*4{c}}
                    & \tikzmark{left}{5} & 9  & \\
                    & 6 & 10 & \\
                    & 7 & 11 & \\
                    & \tikzmark{right}{8} & 12 &\\
                    \end{array}
                    \right] \\
                \Highlight[second]
                s &= \left[ \begin{array}{*4{c}}
                    & \tikzmark{left}{70} \tikzmark{right}{} & & \\
                    & & &
                    \end{array}
                    \right] \\
                \Highlight[third]
            \end{align*}
        \end{frame}

        \begin{frame}
            \frametitle{Batched Multi-class Logistic Regression}
            \begin{align*}
                f &= \left[ \begin{array}{*6{c}}
                    & \tikzmark{left}{1} & 2 & 3 & \tikzmark{right}{4} & \\
                    & 13 & 14 & 15 & 16 &
                    \end{array}
                    \right] \\
                \Highlight[first]
                W &= \left[ \begin{array}{*4{c}}
                    & 5 & \tikzmark{left}{9} & \\
                    & 6 & 10 & \\
                    & 7 & 11 & \\
                    & 8 & \tikzmark{right}{12} &
                    \end{array}
                    \right] \\
                \Highlight[second]
                s &= \left[ \begin{array}{*4{c}}
                    & 70 & \tikzmark{left}{110} \tikzmark{right}{} & \\
                    & & &
                    \end{array}
                    \right] \\
                \Highlight[third]
            \end{align*}
        \end{frame}

        \begin{frame}
            \frametitle{Batched Multi-class Logistic Regression}
            \begin{align*}
                f &= \left[ \begin{array}{*6{c}}
                    & 1 & 2 & 3 & 4 & \\
                    & \tikzmark{left}{13} & 14 & 15 & \tikzmark{right}{16} &
                    \end{array}
                    \right] \\
                \Highlight[first]
                W &= \left[ \begin{array}{*4{c}}
                    & \tikzmark{left}{5} & 9  & \\
                    & 6 & 10 & \\
                    & 7 & 11 & \\
                    & \tikzmark{right}{8} & 12 &
                    \end{array}
                    \right] \\
                \Highlight[second]
                s &= \left[ \begin{array}{*4{c}}
                    & 70 & 110 & \\
                    & \tikzmark{left}{382}\tikzmark{right}{} & &
                    \end{array}
                    \right] \\
                \Highlight[third]
            \end{align*}
        \end{frame}

        \begin{frame}
            \frametitle{Batched Multi-class Logistic Regression}
            \begin{align*}
                f &= \left[ \begin{array}{*6{c}}
                    & 1 & 2 & 3 & 4 & \\
                    & \tikzmark{left}{13} & 14 & 15 & \tikzmark{right}{16} &
                    \end{array}
                    \right] \\
                \Highlight[first]
                W &= \left[ \begin{array}{*4{c}}
                    & 5 & \tikzmark{left}{9} & \\
                    & 6 & 10 & \\
                    & 7 & 11 & \\
                    & 8 & \tikzmark{right}{12} &
                    \end{array}
                    \right] \\
                \Highlight[second]
                s &= \left[ \begin{array}{*4{c}}
                    & 70 & 110 & \\
                    & 382 & \tikzmark{left}{614} \tikzmark{right}{}&
                    \end{array}
                    \right] \\
                \Highlight[third]
            \end{align*}
        \end{frame}


    \end{subsection} % What

    \begin{subsection}{Why Batching?}

        \begin{frame}
            \frametitle{Full Gradient Descent}

        \end{frame}

        \begin{frame}
            \frametitle{Stochastic Gradient Descent}

        \end{frame}

        \begin{frame}
            \frametitle{Mini-Batched Gradient Descent}

        \end{frame}

        \begin{frame}
            \frametitle{Gradient Accumulation?}

        \end{frame}

        \begin{frame}
            \frametitle{Speed}

        \end{frame}

    \end{subsection} % Why

\end{section} % Motivation

% \begin{section}{Batching is hard in NLP}

% \end{section} % Hard in NLP

\begin{section}{Batching can Introduce Errors}

    \begin{subsection}{Mean Pooling}
        \begin{frame}
            \frametitle{Mean Pooling}
            \begin{itemize}
            \item[]<1->$
                \left[ \begin{array}{*6{c}}
                    1 & 10 & 8 & 17 & 13 & 17
                \end{array} \right] = \frac{66}{6} = 11.0
            $
            \item[]
            \item[]<2->$
                \left[ \begin{array}{*4{c}}
                    22 & 24 & 9 & 13
                \end{array} \right] = \frac{68}{4} = 17.0
            $
            \item[]
            \item[]<3->$
                \left[ \begin{array}{*6{c}}
                    5 & 4 & 8 & 9 & 10 & 34 \\
                    6 & 3 & 1 & 4 & 0 & 0
                \end{array} \right] =
                \left[ \begin{array}{c}
                    \frac{66}{6} \\
                    \frac{68}{6}
                \end{array} \right] =
                \left[ \begin{array}{c}
                    11.0 \\
                    11.\bar{3}
                \end{array} \right]
            $
            \end{itemize}
        \end{frame}

        \begin{frame}
            \frametitle{Mean Pooling}

        \end{frame}

\begin{frame}[fragile]
    \frametitle{Mean Pooling}

    \begin{pythoncode}
            >>> x
            array([[ 1, 10,  8, 17, 13, 17],
                   [22, 24,  9, 13,  0,  0]])
            >>> np.mean(x, axis=1)
            array([11.        , 11.33333333])
            >>> lengths
            array([6, 4])
            >>> np.sum(x, axis=1) / lengths
            array([11., 17.])
    \end{pythoncode}

\end{frame}

    \end{subsection} % Mean Pooling

    \begin{subsection}{Token Level Losses}

        \begin{frame}
            \frametitle{Cross Entropy Loss}
            \begin{align*}
                T &\coloneqq \text{length of example} \\
                V &\coloneqq \text{number of possible labels} \\
                s &\in \R^{T \text{ x } V} \\
                l &\in \N^T \\
                J &= - \sum_{i}^{T} s_i[l_i] \\
            \end{align*}

        \end{frame}

        \begin{frame}
            \frametitle{Cross Entropy Loss}
            \begin{align*}
                s &= \left[ \begin{array}{*5{c}}
                    -0.34 & 0.1 & -0.11 & \cdots & 0.001 \\
                    0.93 & -8.88 & -0.39 & \cdots & 0.12 \\
                    & & \vdots & & \\
                    -0.45 & 0.23 & 1.1 & \cdots & -0.3
                \end{array} \right] \\
                l &= \left[ \begin{array}{*5{c}}
                    2 & 1 & 10 & 5 & 1
                \end{array} \right]
            \end{align*}
        \end{frame}
    \end{subsection} % Token Level Losses

    % \begin{subsection}{Attention}

    % \end{subsection} % Attention

    % \begin{subsection}{Conditional Random Field}

    % \end{subsection} % CRF

    % \begin{subsection}{Beam Search}

    % \end{subsection} % Beam Search

    % \begin{subsection}{Convolution 1D}

    % \end{subsection} % Convolution

    % \begin{subsection}{Max Pooling}

    % \end{subsection}

\end{section}

\end{document}
